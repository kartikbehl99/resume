\documentclass[letterpaper]{main} 
\usepackage{fixltx2e}
\usepackage{hyperref}
\usepackage{color}
\usepackage[document]{ragged2e}

\begin{document}

%----------------------------------------------------------------------------------------
%	TITLE SECTION
%----------------------------------------------------------------------------------------

\begin{flushleft}
\Huge \textbf{Kartik Behl}\\*
\begin{small}
\vspace{-8mm}
\normalsize \textbf{\urlstyle{same}{\footnotesize \\ \textbf{\href{https://www.linkedin.com/in/kartik-b-b46a3484/}{LinkedIn}| \href{https://github.com/kartikbehl99}{GitHub} | \href{tel:9990290880}{Phone} | \href{mailto:kartikbehl99@gmail.com}{Email}}}}
\end{small}
\end{flushleft}

%----------------------------------------------------------------------------------------
%	LEFT COLUMN
%----------------------------------------------------------------------------------------

\begin{minipage}[t]{0.29\textwidth}

\section{Skills}
\subsection{Languages}
Python \textbullet{}C++ \textbullet{} JavaScript\textbullet{}Ruby
\sectionspace
\subsection{Frameworks}
 Flask \textbullet{} Angular\textbullet{} Ruby on Rails
\sectionspace
\subsection{Database}
 PostgreSQL \textbullet{} MySQL\textbullet{} DynamoDB
\sectionspace
\subsection{Tools and Technologies}
 Git \textbullet{} AWS\textbullet{} Selenium\textbullet{} Ansible\textbullet{} HTML\textbullet{} CSS
\sectionspace


\section{Education} 
\subsection{Maharaja Agrasen Institute of Technology}
\descript{Bachelor of Technology}
\location{2017-Present | CGPA: 8.7}


\section{CourseWork}
\sectionspace
\subsection{Undergraduate}
\sectionspace
Algorithms and Data Structures \\
Databases\\
Computer Networks \\
Cloud Computing \\
Operating Systems \\
Cryptography and Network Security \\
\sectionspace
\subsection{Online Coursework}
\sectionspace
Algorithms and Data Structures \\
Cryptography and Information Theory \\
Full stack Web development \\
AWS Fundamentals \\


\section{Achievements}
\begin{itemize}
\item Winner among 50 teams at HackBPIT, a 24 hr hackathon.
\item 5 stars in Problem Solving on Hackerrank.
\end{itemize}
\sectionspace 

\end{minipage} 
\hfill
%
%----------------------------------------------------------------------------------------
%	RIGHT COLUMN
%----------------------------------------------------------------------------------------
%
\begin{minipage}[t]{0.66\textwidth} 

\section{Experience}
\runsubsection{Software Engineer} |\descript{\small Coding Ninjas}
\location{July 2020 - Present | Delhi, India}
\begin{tightitemize}
\sectionspace
\item Worked as a full stack developer.
\item Developed end-to-end features using Angular and Ruby on Rails. \\
\item Implemented caching using redis thereby reducing API response time. \\
\item Optimized certain database queries by removing N+1 query problem where feasible.
\end{tightitemize}
\sectionspace 

\runsubsection{Software Engineer Intern} |\descript{\small OpsLyft}
\location{June 2019 – June 2020 | Noida, India}
\begin{tightitemize}
\item Primarily worked as a backend developer.\\
\item Developed and optimized the backend of the product. \\
\item Reduced query time from \textasciitilde2 minutes to \textasciitilde10 seconds by implementing caching and optimizing SQL queries.
\item Automated deployment process using Ansible. \\
\item Developed Tech products for marketing team such as Sales navigator scraper, crunchbase scraper etc. \\
\end{tightitemize}
\sectionspace 

\runsubsection{Teaching Assistant} |\descript{\small Coding Ninjas}
\location{Jan 2018 - April 2018 | Delhi, India}
\begin{tightitemize}
\item Worked as a TA for Data structures and Algorithms.
\item Took doubt sessions and explained various topics. \\
\item Extensively debugged codes. \\
\end{tightitemize}
\sectionspace 


\section{Projects}

\runsubsection{SSH Manager} |\descript{\small Personal Project}
\begin{tightitemize}
\item it is a CLI tool for managing SSH keys of an organizaধon and SSH to
the server. It encrypts all the keys with a password known to the user
only and isn’t stored any where else. The tool also allows controlling
who has access to a particular SSH key. \\
\item Technologies: Flask, AWS EC2, AWS DynamoDB, SSH, Ansible.\\
\end{tightitemize}
\sectionspace 

\runsubsection{URL Shortener} |\descript{\small Personal Project}
\begin{tightitemize}
\item A URL shortener capable of generating \textasciitilde3 trillion unique URLs without the need of checking for collisions. \\
\item Technologies: Angular, Flask, Nginx, EC2, MongoDB. \\
\end{tightitemize}
\sectionspace 

\runsubsection{Automated User Behaviour Mapping} |\descript{\small Personal Project}
\begin{tightitemize}
\item A javascript and python based app which records all the actions performed by a user on a website in real time and then let's you see how the user operated the website/webpage by replicating the actions in a browser window. \\
\item The process of replication is entirely automated using selenium. \\
\item Technologies: JavaScript, Python, Angular, SocketIO, Selenium.
\end{tightitemize}
\sectionspace 

\runsubsection{AWS Cost Dashboard} |\descript{\small Developed at OpsLyft}
\begin{tightitemize}
\item AWS Cost Dashboard is aimed at providing near real time cost analytics and usage of various AWS services. \\
\item It also shows various real time usage statistics such as CPU Utilization, Memory Utilization, Volume used etc. \\
\item Technologies: Python, Flask, ReactJs, AWS(EC2, Redshift, S3,Lambda, CloudFormation etc.), boto3.
\end{tightitemize}

\end{minipage}

\end{document}
